%---- Required Packages and Functions ----

\documentclass[a4paper,11pt]{article}
\usepackage{latexsym}
\usepackage{xcolor}
\usepackage{float}
\usepackage{ragged2e}
\usepackage[empty]{fullpage}
\usepackage{wrapfig}
\usepackage{lipsum}
\usepackage{tabularx}
\usepackage{titlesec}
\usepackage{geometry}
\usepackage{marvosym}
\usepackage{verbatim}
\usepackage{enumitem}
\usepackage[hidelinks]{hyperref}
\usepackage{fancyhdr}
\usepackage{fontawesome5}
\usepackage{multicol}
\usepackage{graphicx}
\usepackage{cfr-lm}
\usepackage[T1]{fontenc}
\setlength{\footskip}{4.08003pt} 
\pagestyle{fancy}
\fancyhf{} % clear all header and footer fields
\fancyfoot{}
\renewcommand{\headrulewidth}{0pt}
\renewcommand{\footrulewidth}{0pt}
\geometry{left=1.4cm, top=1cm, right=1.4cm, bottom=1cm}
% Adjust margins
%\addtolength{\oddsidemargin}{-0.5in}
%\addtolength{\evensidemargin}{-0.5in}
%\addtolength{\textwidth}{1in}
\usepackage[most]{tcolorbox}
\tcbset{
	frame code={}
	center title,
	left=0pt,
	right=0pt,
	top=0pt,
	bottom=0pt,
	colback=gray!20,
	colframe=white,
	width=\dimexpr\textwidth\relax,
	enlarge left by=-2mm,
	boxsep=4pt,
	arc=0pt,outer arc=0pt,
}

\urlstyle{same}

\raggedright
\setlength{\footskip}{4.08003pt}

% Sections formatting
\titleformat{\section}{
  \vspace{-4pt}\scshape\raggedright\large
}{}{0em}{}[\color{black}\titlerule \vspace{-7pt}]

%-------------------------
% Custom commands
\newcommand{\resumeItem}[2]{
  \item{
    \textbf{#1}{\hspace{0.5mm}#2 \vspace{-0.5mm}}
  }
}

\newcommand{\resumePOR}[3]{
\vspace{0.5mm}\item
    \begin{tabular*}{0.97\textwidth}[t]{l@{\extracolsep{\fill}}r}
        \textbf{#1}\hspace{0.3mm}#2 & \textit{\small{#3}} 
    \end{tabular*}
    \vspace{-2mm}
}

\newcommand{\resumeSubheading}[4]{
\vspace{0.5mm}\item
    \begin{tabular*}{0.98\textwidth}[t]{l@{\extracolsep{\fill}}r}
        \textbf{#1} & \textit{\footnotesize{#4}} \\
        \textit{\footnotesize{#3}} &  \footnotesize{#2}\\
    \end{tabular*}
    \vspace{-2.4mm}
}

\newcommand{\resumeProject}[4]{
\vspace{0.5mm}\item
    \begin{tabular*}{0.98\textwidth}[t]{l@{\extracolsep{\fill}}r}
        \textbf{#1} & \textit{\footnotesize{#3}} \\
        \footnotesize{\textit{#2}} & \footnotesize{#4}
    \end{tabular*}
    \vspace{-2.4mm}
}

\newcommand{\resumeSubItem}[2]{\resumeItem{#1}{#2}\vspace{-4pt}}

% \renewcommand{\labelitemii}{$\circ$}
\renewcommand{\labelitemi}{$\vcenter{\hbox{\tiny$\bullet$}}$}

\newcommand{\resumeSubHeadingListStart}{\begin{itemize}[leftmargin=*,labelsep=0mm]}
\newcommand{\resumeHeadingSkillStart}{\begin{itemize}[leftmargin=*,itemsep=1.7mm, rightmargin=2ex]}
\newcommand{\resumeItemListStart}{\begin{justify}\begin{itemize}[leftmargin=3ex, rightmargin=2ex, noitemsep,labelsep=1.2mm,itemsep=0mm]\small}

\newcommand{\resumeSubHeadingListEnd}{\end{itemize}\vspace{2mm}}
\newcommand{\resumeHeadingSkillEnd}{\end{itemize}\vspace{-2mm}}
\newcommand{\resumeItemListEnd}{\end{itemize}\end{justify}\vspace{-2mm}}
\newcommand{\cvsection}[1]{%
\vspace{2mm}
\begin{tcolorbox}
    \textbf{\large #1}
\end{tcolorbox}
    \vspace{-4mm}
}

\newcolumntype{L}{>{\raggedright\arraybackslash}X}%
\newcolumntype{R}{>{\raggedleft\arraybackslash}X}%
\newcolumntype{C}{>{\centering\arraybackslash}X}%
%---- End of Packages and Functions ------

%-------------------------------------------
%%%%%%  CV STARTS HERE  %%%%%%%%%%%
%%%%%% DEFINE ELEMENTS HERE %%%%%%%
\newcommand{\name}{MohammadReza Fazli} % Your Name
\newcommand{\course}{Master of Electrical Engineering} % Your Program
\newcommand{\phone}{936 018 7828} % Your Phone Number
\newcommand{\emaila}{Fazli.m9929@gmail.com} %Email 1
\newcommand{\emailb}{Fazli.m@ut.ac.ir} %Email 2




\begin{document}
\fontfamily{cmr}\selectfont
%----------HEADING-----------------


\parbox{2.6cm}{%
\includegraphics[width=2.35cm,clip]{my_pic.png}
}
\parbox{\dimexpr\linewidth-2.9cm\relax}{
\begin{tabularx}{\linewidth}{L r} \\
  \textbf{\Large \name} & {\raisebox{0.0\height}{\footnotesize \faPhone}\ +98 \phone}\\
  \course & \href{mailto:\emaila}{\raisebox{0.0\height}{\footnotesize \faEnvelope}\ {\emaila}} \\
  {School of ECE} &  \href{mailto:\emailb}{\raisebox{0.0\height}{\footnotesize \faEnvelope}\ {\emailb}}\\
  {University of Tehran} &  \href{https://github.com/fazli-m9929/}{\raisebox{0.0\height}{\footnotesize \faGithub}\ {GitHub Profile}} \\
\end{tabularx}
}


%-----------Technical skills-----------------
\section{\textbf{About me}}
  \justifying
  \begin{itemize}[leftmargin=0.1in, label={}]
    \item{
      I am a eager to work in data Science field and currently studying to get my masters degree in electrical engineering. I also have a strong interest in image analysis, natural language processing, and computer vision.
      I have a proven track record of success in developing and deploying innovative machine learning solutions to real-world problems.

      In my role as a data scientist, I have worked on a variety of projects, including developing deep learning algorithms to improve the accuracy and efficiency of medical image analysis, extract insights from natural language data, and automate tasks.
      I have also developed a deep understanding of the deep learning pipeline, from data collection and preparation to model training and evaluation.

      I am also a skilled communicator and team player.
      I am always eager to learn new things and share my knowledge with others.
      I am confident that I have the skills and experience necessary to be a valuable asset to your team.

      I am excited about the opportunity to contribute to your company's mission and make a positive impact.
      I look forward to learning more about your open position and discussing how my deep learning skills and experience can be of benefit to your team.

      I hope this is more to your liking. Please let me know if you have any other questions or requests.
    }
 \end{itemize}
 \vspace{-16pt}


%-----------EDUCATION-----------
\section{\textbf{Education}}
  \resumeSubHeadingListStart
    \resumeSubheading
      {Bu-Ali Sina University}{CGPA: 18.56}
      {Bachelor of Science in Electrical Engineering}{2021}
    \resumeSubheading
      {University of Tehran}{CGPA: 18.63}
      {Master of Science in Electrical Engineering}{2024}
  \resumeSubHeadingListEnd
\vspace{-5.5mm}
%


%-----------Master-----------------
\section{\textbf{Master Thesis}}
  \resumeSubHeadingListStart

      \resumeProject
      {Digital Pulse Shaping Filters for Fast Particle-Detector Spectroscopy} %Project Name
      {Developed digital pulse shaping filters for fast particle-detector spectroscopy using OOP Python}
      {University of Tehran}

      \resumeItemListStart
        \item {I designed and implemented digital pulse shaping filters for fast particle-detector spectroscopy}
        \item {The filters were designed to improve the energy resolution and discrimination capability of the detector system and implemented using Object-Oriented Programming in Python to test the performance of filters}
        \item {Apart from simulation, real data was collected using HPGe detector to further evaluate the performance}
        \item {I used a machine learning technique after shaping the signals to further improve the performance of the detector system and a paper is being written to describe this work in more detail}
      \resumeItemListEnd

  \resumeSubHeadingListEnd
\vspace{-8.5mm}


%-----------Online courses-----------
\section{\textbf{Online courses}}
  \resumeSubHeadingListStart

    \resumeSubheading
      {Python Developer}{}
      {Sololearn - Verify at: \href{https://www.sololearn.com/certificates/CC-GO0ECLZV/}{https://www.sololearn.com/certificates/CC-GO0ECLZV/}}{2023}
    \resumeSubheading
      {Databases and SQL for Data Science with Python}{20 hours}
      {Coursera - Verify at: \href{https://coursera.org/verify/KWDA3ZNARQJF/}{https://coursera.org/verify/KWDA3ZNARQJF/}}{2024}
    \resumeSubheading
      {The Git \& GitHub Bootcamp: The Compelete-Practical Guide}{6.5 hours}
      {Udemy - Verify at: \href{https://ude.my/UC-4b07695d-68b3-4f11-921f-873815378914/}{https://ude.my/UC-4b07695d-68b3-4f11-921f-873815378914/}}{2024}

  \resumeSubHeadingListEnd
\vspace{-5.5mm}
%

%-----------EXPERIENCE-----------------
\section{\textbf{Experience}}
  \resumeSubHeadingListStart
    \resumeSubheading
      {Fater Sharif}{Tehran}
      {FPGA and VHDL intern}{2021}
      \vspace{-2.0mm}
      \resumeItemListStart
        \item {Design a VHDL code for an error correction transmission system}
        \item {Implemented a BCH error correction encoder and decoder in a DVB-S2 transmission standard}
      \resumeItemListEnd
    
  \resumeSubHeadingListEnd
\vspace{-8.5mm}


%-----------PROJECTS-----------------
\section{\textbf{Personal Projects}}
\resumeSubHeadingListStart

    \resumeProject
      {EyeLesionSegmentation with EfficientNetV3} %Project Name
      {Design and implementation of a U-Net using EfficientNetV3 as encoder block for segmenting ill areas of the eye} %Project Name, Location Name
      {2023} %Event Dates

      \resumeItemListStart
        \item {The goal of this project was to design and implement a U-Net using EfficientNetV3 as the encoder block to segment ill areas of the eye. EfficientNetV3 is a state-of-the-art image classification model that is known for its efficiency and accuracy. U-Net is a deep learning architecture that is commonly used for image segmentation tasks.}
        \item {The project involved 
          Collecting a dataset of ill and normal eye images,
          Preprocessing the images by resizing them and normalizing their pixel values,
          Designing a U-Net architecture with EfficientNetV3 as the encoder block and a decoder block that was created from scratch,
          Implementing the U-Net architecture in PyTorch,
          Training the U-Net model on the collected dataset and
          Evaluating the performance of the trained model on a held-out test set.
        }
        \item {Tools and Technologies Used are: PyTorch, EfficientNetV3 and U-Net.}
        \item {The parameter count of the EfficientNetV3 encoder block was 13M. The decoder block had the same parameter count as the encoder block, resulting in a total parameter count of 26M.}
        \item { The model was trained on Google Colab due to GPU usage.}
      \resumeItemListEnd
    \vspace{-2mm}
    
    \resumeProject
      {ResNet for Medical Image Classification} %Project Name
      {Implementation of "Deep Learning in Image Classification using ResNet Variants for Detection of Colorectal Cancer"} %Project Name, Location Name
      {2022} %Event Dates

      \resumeItemListStart
        \item {The goal of this project was to implement a deep learning model using ResNet variants for the detection of colorectal cancer. The model was trained and evaluated on a publicly available dataset of colorectal cancer images (Warwick-QU).}
        \item {
          The project involved the following steps: 
          Collecting cancer images and Preprocessing the images by resizing them and normalizing their pixel values.
          Implementing different ResNet18 and ResNet50 variants in PyTorch from torchvision library.
          Training the ResNet models and Evaluating the performance.
        }
        \item {Tools and Technologies Used: PyTorch and ResNet variants}
      \resumeItemListEnd
    \vspace{-2mm}
    

    \resumeProject
    {Air Pollution Prediction using CNN-LSTM} %Project Name
    {Implementation of "Air pollution prediction in smart city, deep learning approach"} %Project Name, Location Name
    {2022} %Event Dates

    \resumeItemListStart
      \item {The goal of this project was to develop a deep learning model to predict air pollution levels in a smart city. The model was trained on a dataset of air pollution data from Beijing}
      \item {
        The project involved the following steps:
        Implementing a CNN-LSTM model in PyTorch.
        Training the model on the collected dataset.
        Evaluating the performance of the trained model on a held-out test set.
      }
      \item {The trained CNN-LSTM model was able to achieve an R-squared score of 91\% for 24-hour predictions and an R-squared score of 88\% for 7-day predictions. This suggests that the model is able to accurately predict air pollution levels for both short-term and long-term horizons.}
      \item {Tools and Technologies Used: PyTorch, CNN-LSTM}
    \resumeItemListEnd
  \vspace{-2mm}
      

  \resumeProject
  {Model for Generating Hand Images from Noise} %Project Name
  {The goal of this project was to develop a Generative Adversarial Network model that could generate images from noise} %Project Name, Location Name
  {2023} %Event Dates

  \resumeItemListStart
    \item {
      The project involved the following steps: 
      Data collection and preprocessing.
      Implementing a GAN model in PyTorch.
      Training the GAN model and evaluating the performance.
    }
    \item {Tools and Technologies Used: PyTorch, Generative Adversarial Network (GAN)}
  \resumeItemListEnd
  \vspace{-2mm}

\resumeSubHeadingListEnd
\vspace{-5.5mm}



%-----------Technical skills-----------------
\section{\textbf{Technical Skills and Interests}}
  \begin{itemize}[leftmargin=0.1in, label={}]
    \small{\item{
      \textbf{Language skill}{: Advanced English} \\
      \textbf{Programming Languages}{: Python, SQL, Embeded system C, VHDL} \\
      \textbf{Developer Tools}{: Visual Studio Code, Git, Jupyter Notebook, Vivado, Circuit Sim applications} \\
      \textbf{Frameworks}{: PyTorch, Open-CV, Pandas, Scikit-learn, Numpy, ...} \\
      \textbf{Cloud/Databases}{: SQLite, MySQL} \\
      \textbf{Soft Skills}{: Communication, Teamwork, Problem-solving, Critical thinking} \\
      \textbf{Coursework}{: Machine learning, Deep learning, Computer vision, Natural language processing, Artificial intelligence} \\
      \textbf{Areas of Interest}{: Medical image analysis, Natural language processing, Computer vision} \\
    }}
 \end{itemize}
 \vspace{-16pt}



% %-----------Positions of Responsibility-----------------
% \section{\textbf{Positions of Responsibility}}
% \vspace{-0.4mm}
% \resumeSubHeadingListStart
% \resumePOR{Position, } % Position
%     {Club or Event} %Club,Event
%     {Position tenure} %Tenure Period
% \resumePOR{Position, } % Position
%     {Club or Event} %Club,Event
%     {Position tenure} %Tenure Period
% \resumeSubHeadingListEnd
% \vspace{-5mm}




% %-----------Achievements-----------------
% \section{\textbf{Achievements}}
% \vspace{-0.4mm}
% \resumeSubHeadingListStart
% \resumePOR{Achievement } % Award
%     {description} % Event
%     {Event dates} %Event Year
    
% \resumePOR{Achievement } % Award
%     {description} % Event
%     {Event dates} %Event Year
% \resumeSubHeadingListEnd
% \vspace{-5mm}



%-------------------------------------------
\end{document}
