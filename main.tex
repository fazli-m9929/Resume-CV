%---- Required Packages and Functions ----

\documentclass[a4paper,11pt]{article}
\usepackage{latexsym}
\usepackage{xcolor}
\usepackage{float}
\usepackage{ragged2e}
\usepackage[empty]{fullpage}
\usepackage{wrapfig}
\usepackage{lipsum}
\usepackage{tabularx}
\usepackage{titlesec}
\usepackage{geometry}
\usepackage{marvosym}
\usepackage{verbatim}
\usepackage{enumitem}
\usepackage[hidelinks]{hyperref}
\usepackage{fancyhdr}
\usepackage{fontawesome5}
\usepackage{multicol}
\usepackage{graphicx}
\usepackage{cfr-lm}
\usepackage[T1]{fontenc}
\setlength{\footskip}{4.08003pt} 
\pagestyle{fancy}
\fancyhf{} % clear all header and footer fields
\fancyfoot{}
\renewcommand{\headrulewidth}{0pt}
\renewcommand{\footrulewidth}{0pt}
\geometry{left=1.4cm, top=1cm, right=1.4cm, bottom=1cm}
% Adjust margins
%\addtolength{\oddsidemargin}{-0.5in}
%\addtolength{\evensidemargin}{-0.5in}
%\addtolength{\textwidth}{1in}
\usepackage[most]{tcolorbox}
\tcbset{
	frame code={}
	center title,
	left=0pt,
	right=0pt,
	top=0pt,
	bottom=0pt,
	colback=gray!20,
	colframe=white,
	width=\dimexpr\textwidth\relax,
	enlarge left by=-2mm,
	boxsep=4pt,
	arc=0pt,outer arc=0pt,
}

\urlstyle{same}

\raggedright
\setlength{\footskip}{4.08003pt}

% Sections formatting
\titleformat{\section}{
  \vspace{-4pt}\scshape\raggedright\large
}{}{0em}{}[\color{black}\titlerule \vspace{-7pt}]

%-------------------------
% Custom commands
\newcommand{\resumeItem}[2]{
  \item{
    \textbf{#1}{\hspace{0.5mm}#2 \vspace{-0.5mm}}
  }
}

\newcommand{\resumePOR}[3]{
\vspace{0.5mm}\item
    \begin{tabular*}{0.97\textwidth}[t]{l@{\extracolsep{\fill}}r}
        \textbf{#1}\hspace{0.3mm}#2 & \textit{\small{#3}} 
    \end{tabular*}
    \vspace{-2mm}
}

\newcommand{\resumeSubheading}[4]{
\vspace{0.5mm}\item
    \begin{tabular*}{0.98\textwidth}[t]{l@{\extracolsep{\fill}}r}
        \textbf{#1} & \textit{\footnotesize{#4}} \\
        \textit{\footnotesize{#3}} &  \footnotesize{#2}\\
    \end{tabular*}
    \vspace{-2.4mm}
}

\newcommand{\resumeProject}[4]{
\vspace{0.5mm}\item
    \begin{tabular*}{0.98\textwidth}[t]{l@{\extracolsep{\fill}}r}
        \textbf{#1} & \textit{\footnotesize{#3}} \\
        \footnotesize{\textit{#2}} & \footnotesize{#4}
    \end{tabular*}
    \vspace{-2.4mm}
}

\newcommand{\resumeSubItem}[2]{\resumeItem{#1}{#2}\vspace{-4pt}}

% \renewcommand{\labelitemii}{$\circ$}
\renewcommand{\labelitemi}{$\vcenter{\hbox{\tiny$\bullet$}}$}

\newcommand{\resumeSubHeadingListStart}{\begin{itemize}[leftmargin=*,labelsep=0mm]}
\newcommand{\resumeHeadingSkillStart}{\begin{itemize}[leftmargin=*,itemsep=1.7mm, rightmargin=2ex]}
\newcommand{\resumeItemListStart}{\begin{justify}\begin{itemize}[leftmargin=3ex, rightmargin=2ex, noitemsep,labelsep=1.2mm,itemsep=0mm]\small}

\newcommand{\resumeSubHeadingListEnd}{\end{itemize}\vspace{2mm}}
\newcommand{\resumeHeadingSkillEnd}{\end{itemize}\vspace{-2mm}}
\newcommand{\resumeItemListEnd}{\end{itemize}\end{justify}\vspace{-2mm}}
\newcommand{\cvsection}[1]{%
\vspace{2mm}
\begin{tcolorbox}
    \textbf{\large #1}
\end{tcolorbox}
    \vspace{-4mm}
}

\newcolumntype{L}{>{\raggedright\arraybackslash}X}%
\newcolumntype{R}{>{\raggedleft\arraybackslash}X}%
\newcolumntype{C}{>{\centering\arraybackslash}X}%
%---- End of Packages and Functions ------

%-------------------------------------------
%%%%%%  CV STARTS HERE  %%%%%%%%%%%
%%%%%% DEFINE ELEMENTS HERE %%%%%%%
\newcommand{\name}{MohammadReza Fazli} % Your Name
\newcommand{\course}{Master of Electrical Engineering} % Your Program
\newcommand{\phone}{936 018 7828} % Your Phone Number
\newcommand{\emaila}{Fazli.m9929@gmail.com} %Email 1
\newcommand{\emailb}{Fazli.m@ut.ac.ir} %Email 2




\begin{document}
\fontfamily{cmr}\selectfont
%----------HEADING-----------------


\parbox{2.6cm}{%
\includegraphics[width=2.35cm,clip]{my_pic.png}
}
\parbox{\dimexpr\linewidth-2.9cm\relax}{
\begin{tabularx}{\linewidth}{L r} \\
  \textbf{\Large \name} & {\raisebox{0.0\height}{\footnotesize \faPhone}\ +98 \phone}\\
  \course & \href{mailto:\emaila}{\raisebox{0.0\height}{\footnotesize \faEnvelope}\ {\emaila}} \\
  {School of ECE} &  \href{mailto:\emailb}{\raisebox{0.0\height}{\footnotesize \faEnvelope}\ {\emailb}}\\
  {University of Tehran} &  \href{https://github.com/fazli-m9929/}{\raisebox{0.0\height}{\footnotesize \faGithub}\ {GitHub Profile}} \\
\end{tabularx}
}


%-----------Technical skills-----------------
\section{\textbf{About me}}
  \justifying
  \begin{itemize}[leftmargin=0.1in, label={}]
    \item{
      I am an enthusiastic Data Engineer with a Master’s degree in Electrical Engineering, currently specializing in large language models (LLMs) at DBA Company. I have a strong interest in data science, particularly in natural language processing, image analysis, and computer vision.
    
      I have a proven track record of developing and deploying innovative machine learning solutions to address real-world challenges. My experience includes creating deep learning algorithms that enhance the accuracy and efficiency of various applications, such as medical image analysis and natural language data extraction.
    
      My expertise extends throughout the deep learning pipeline, encompassing data collection, preparation, model training, and evaluation. I pride myself on my ability to communicate complex concepts clearly and effectively, making me a skilled collaborator in team settings.
    
      I am committed to continuous learning and enjoy sharing knowledge with others. I am confident that my skills and experience will be valuable assets to your team.
    
      I am excited about the opportunity to contribute to your company's mission and make a positive impact. I look forward to discussing how my expertise in LLMs and Deep learning models can benefit your team.
    }
    
 \end{itemize}
 \vspace{-16pt}


%-----------EDUCATION-----------
\section{\textbf{Education}}
  \resumeSubHeadingListStart
    \resumeSubheading
      {Bu-Ali Sina University}{CGPA: 18.56}
      {Bachelor of Science in Electrical Engineering}{2021}
    \resumeSubheading
      {University of Tehran}{CGPA: 18.63}
      {Master of Science in Electrical Engineering}{2024}
  \resumeSubHeadingListEnd
\vspace{-5.5mm}
%


%-----------Master-----------------
\section{\textbf{Master Thesis}}
  \resumeSubHeadingListStart

      \resumeProject
      {Digital Pulse Shaping Filters for Fast Particle-Detector Spectroscopy} %Project Name
      {Developed digital pulse shaping filters for fast particle-detector spectroscopy using OOP Python}
      {University of Tehran}

      \resumeItemListStart
        \item {I designed and implemented digital pulse shaping filters for fast particle-detector spectroscopy}
        \item {The filters were designed to improve the energy resolution and discrimination capability of the detector system and implemented using Object-Oriented Programming in Python to test the performance of filters}
        \item {Apart from simulation, real data was collected using HPGe detector to further evaluate the performance}
        \item {I used a machine learning technique after shaping the signals to further improve the performance of the detector system and a paper is being written to describe this work in more detail}
      \resumeItemListEnd

  \resumeSubHeadingListEnd
\vspace{-8.5mm}


%-----------Online courses-----------
\section{\textbf{Online courses}}
  \resumeSubHeadingListStart

    \resumeSubheading
      {Python Developer}{}
      {Sololearn - Verify at: \href{https://www.sololearn.com/certificates/CC-GO0ECLZV/}{https://www.sololearn.com/certificates/CC-GO0ECLZV/}}{2023}
    \resumeSubheading
      {The Git \& GitHub Bootcamp: The Compelete-Practical Guide}{6.5 hours}
      {Udemy - Verify at: \href{https://ude.my/UC-4b07695d-68b3-4f11-921f-873815378914/}{https://ude.my/UC-4b07695d-68b3-4f11-921f-873815378914/}}{2024}
    \resumeSubheading
      {Databases and SQL for Data Science with Python}{20 hours}
      {Coursera - Verify at: \href{https://coursera.org/verify/KWDA3ZNARQJF/}{https://coursera.org/verify/KWDA3ZNARQJF/}}{2024}
    \resumeSubheading
      {Mastering Natural Language Processing: A Comprehensive Guide}{1 hours}
      {Udemy - Verify at: \href{https://ude.my/UC-b94d4a05-89ae-4a20-aea3-f50529a2275f/}{https://ude.my/UC-b94d4a05-89ae-4a20-aea3-f50529a2275f/}}{2024}

    \resumeSubheading
      {Build REST APIs with Python, Django REST Framework: Web API}{10 hours}
      {Udemy - Verify at: \href{https://www.udemy.com/certificate/UC-676611f9-1eb5-4771-8027-4bcf93b1438d/}{https://www.udemy.com/certificate/UC-676611f9-1eb5-4771-8027-4bcf93b1438d/}}{2024}

  \resumeSubHeadingListEnd
\vspace{-5.5mm}
%

%-----------EXPERIENCE-----------------
\section{\textbf{Experience}}
  \resumeSubHeadingListStart

    \resumeSubheading
      {Dadeh Pardazan Bonyan Ava (DBA)}{Tehran}
      {Data Engineer}{2024 - Present}
      \vspace{-2.0mm}
      \resumeItemListStart
        \item {Developed a chatbot using Retrieval-Augmented Generation (RAG) techniques, integrating multiple services to enhance user interaction.}
        \item {Implemented services for extracting meaningful text from various input formats and developed database handlers for Milvus (vector database) and Elasticsearch (NoSQL database) to store data and its metadata in sync.}
        \item {Designed and implemented object-oriented Python classes to manage database interactions and service functionality.}
        \item {Utilized Celery with task workers to optimize asynchronous task processing and improve performance.}
        \item {Managed APIs using Django with PostgreSQL as the primary database, ensuring robust data handling and retrieval.}
        \item {Deployed databases using Docker Compose for containerized environments, facilitating streamlined development and production workflows on Ubuntu Linux.}
    \resumeItemListEnd

    \resumeSubheading
      {Fater Sharif}{Tehran}
      {FPGA and VHDL intern}{2021}
      \vspace{-2.0mm}
      \resumeItemListStart
        \item {Design a VHDL code for an error correction transmission system}
        \item {Implemented a BCH error correction encoder and decoder in a DVB-S2 transmission standard}
      \resumeItemListEnd
    
  \resumeSubHeadingListEnd
\vspace{-8.5mm}


%-----------PROJECTS-----------------
\section{\textbf{Personal Projects}}
\resumeSubHeadingListStart

    \resumeProject
      {Eye Movement Tracking Model} % Project Name
      {Development of a model to track eye movement using ResNet50 as the backbone} % Project Name, Location Name
      {2024} % Event Dates

      \resumeItemListStart
        \item {The goal of this project was to track eye movement by locating the eye and its internal corner using a ResNet50 backbone as a regressor.}
        \item {A dataset was collected by recording 10 to 20 seconds of eye movement, cropping the eye images using MediaPipe, and labeling the cropped images with pixel locations and the radius of the iris.}
        \item {The model predicts pixel locations and the radius of the iris, allowing for the creation of a circle around the iris and a line connecting the internal corner to the center to visualize eye movement.}
        \item {Tools and Technologies Used: PyTorch, MediaPipe, OpenCV.}
      \resumeItemListEnd
      \vspace{-2mm}

    \resumeProject
      {EyeLesionSegmentation with EfficientNetV3} %Project Name
      {Design and implementation of a U-Net using EfficientNetV3 as encoder block for segmenting ill areas of the eye} %Project Name, Location Name
      {2023} %Event Dates

      \resumeItemListStart
        \item {The goal of this project was to design and implement a U-Net using EfficientNetV3 as the encoder block to segment ill areas of the eye. EfficientNetV3 is a state-of-the-art image classification model that is known for its efficiency and accuracy. U-Net is a deep learning architecture that is commonly used for image segmentation tasks.}
        \item {The project involved 
          Collecting a dataset of ill and normal eye images,
          Preprocessing the images by resizing them and normalizing their pixel values,
          Designing a U-Net architecture with EfficientNetV3 as the encoder block and a decoder block that was created from scratch,
          Implementing the U-Net architecture in PyTorch,
          Training the U-Net model on the collected dataset and
          Evaluating the performance of the trained model on a held-out test set.
        }
        \item {Tools and Technologies Used are: PyTorch, EfficientNetV3 and U-Net.}
        \item {The parameter count of the EfficientNetV3 encoder block was 13M. The decoder block had the same parameter count as the encoder block, resulting in a total parameter count of 26M.}
        \item { The model was trained on Google Colab due to GPU usage.}
      \resumeItemListEnd
    \vspace{-2mm}
    
    \resumeProject
      {ResNet for Medical Image Classification} %Project Name
      {Implementation of "Deep Learning in Image Classification using ResNet Variants for Detection of Colorectal Cancer"} %Project Name, Location Name
      {2022} %Event Dates

      \resumeItemListStart
        \item {The goal of this project was to implement a deep learning model using ResNet variants for the detection of colorectal cancer. The model was trained and evaluated on a publicly available dataset of colorectal cancer images (Warwick-QU).}
        \item {
          The project involved the following steps: 
          Collecting cancer images and Preprocessing the images by resizing them and normalizing their pixel values.
          Implementing different ResNet18 and ResNet50 variants in PyTorch from torchvision library.
          Training the ResNet models and Evaluating the performance.
        }
        \item {Tools and Technologies Used: PyTorch and ResNet variants}
      \resumeItemListEnd
    \vspace{-2mm}

  \resumeProject
  {Model for Generating Hand Images from Noise} %Project Name
  {The goal of this project was to develop a Generative Adversarial Network model that could generate images from noise} %Project Name, Location Name
  {2023} %Event Dates

  \resumeItemListStart
    \item {
      The project involved the following steps: 
      Data collection and preprocessing.
      Implementing a GAN model in PyTorch.
      Training the GAN model and evaluating the performance.
    }
    \item {Tools and Technologies Used: PyTorch, Generative Adversarial Network (GAN)}
  \resumeItemListEnd
  \vspace{-2mm}

\resumeSubHeadingListEnd
\vspace{-8.5mm}



%-----------Technical skills-----------------
\section{\textbf{Technical Skills and Interests}}
  \begin{itemize}[leftmargin=0.1in, label={}]
    \small{\item{
      \textbf{Language Skills}{: Advanced English} \\
      \textbf{Programming Languages}{: Python, SQL, Embedded C, VHDL} \\
      \textbf{Developer Tools}{: Visual Studio Code, Git, Jupyter Notebook, Docker, RabbitMQ, Celery, Bash scripting, Vivado, Circuit Simulation applications} \\
      \textbf{Frameworks and Libraries}{: PyTorch, OpenCV, Pandas, Scikit-learn, Numpy, Django, LangChain, python-docx, MediaPipe} \\
      \textbf{Machine Learning and NLP Tools}{: Sentence Transformers, CNNs, Transformers, Large Language Models (LLMs), Retrieval-Augmented Generation (RAG)} \\
      \textbf{Database and Cloud}{: PostgreSQL, SQLite, MySQL, Milvus (vector database), Elasticsearch (NoSQL database), Docker Compose} \\
      \textbf{Soft Skills}{: Communication, Teamwork, Problem-solving, Critical Thinking, Adaptability, Technical Documentation} \\
      \textbf{Coursework}{: Machine Learning, Deep Learning, Computer Vision, Natural Language Processing, Artificial Intelligence, Data Engineering} \\
      \textbf{Areas of Interest}{: Large Language Models (LLMs), Medical Image Analysis, Natural Language Processing, Computer Vision, Retrieval-Augmented Generation (RAG)} \\
    }}
 \end{itemize}
 \vspace{-16pt}


%-----------PUBLICATIONS-----------------
\section{\textbf{Publications}}
  \resumeSubHeadingListStart

    \resumeSubheading
    {RESHAPE: Resolution and Efficiency in Spectroscopy Harnessing Adaptive Processor - ... }{Under Review}
    {IEEE Transactions on Instrumentation and Measurement}{Submitted}
    \resumeItemListStart
      \item{Introduced a configurable pulse processing system designed for high-resolution energy spectrum extraction from radiation detector signals in spectroscopy applications. The system utilizes a pulse shaper to fine-tune the shaping time constant, enhancing peak resolution and reducing Full Width at Half Maximum (FWHM).}
      \item{A novel post-processing technique employing the Parzen window is proposed to improve the spectrum and mitigate noise effects, achieving comparable results with significantly fewer samples. Experimental validations with Cobalt-60, Cesium-137, and Europium-152 demonstrate the system's robustness and precision, highlighting its potential for advancing gamma-ray spectroscopy systems.}
    \resumeItemListEnd
    \vspace{-2mm}

    \resumeSubheading
      {Scaled CR-(RC)\textsuperscript{n} Digital Filter Design for Precision Pulse Processing in Spectroscopy Applications}{2024}
      {32nd International Conference on Electrical Engineering (ICEE), IEEE}{}
      \resumeItemListStart
        \item{Presented a novel digital pulse shaping approach for gamma-ray and x-ray spectroscopy using CR and RC filters, targeting challenges with exponential decay inputs. The design introduces a scaling factor to ensure a consistent 1-to-1 amplitude ratio between input and output, and a post-processing step to correct undershoot issues.}
        \item{The approach achieves clean, semi-Gaussian pulse shapes, enhancing accuracy in subsequent analysis. Experimental results with an HPGe detector validate the effectiveness of this filter design in providing accurate energy measurements essential for spectroscopic applications.}
      \resumeItemListEnd
      \vspace{-2mm}
  \resumeSubHeadingListEnd




% %-----------Positions of Responsibility-----------------
% \section{\textbf{Positions of Responsibility}}
% \vspace{-0.4mm}
% \resumeSubHeadingListStart
% \resumePOR{Position, } % Position
%     {Club or Event} %Club,Event
%     {Position tenure} %Tenure Period
% \resumePOR{Position, } % Position
%     {Club or Event} %Club,Event
%     {Position tenure} %Tenure Period
% \resumeSubHeadingListEnd
% \vspace{-5mm}


% %-----------Achievements-----------------
% \section{\textbf{Achievements}}
% \vspace{-0.4mm}
% \resumeSubHeadingListStart
% \resumePOR{Achievement } % Award
%     {description} % Event
%     {Event dates} %Event Year
    
% \resumePOR{Achievement } % Award
%     {description} % Event
%     {Event dates} %Event Year
% \resumeSubHeadingListEnd
% \vspace{-5mm}



%-------------------------------------------
\end{document}
